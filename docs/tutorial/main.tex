\documentclass[11pt]{article}
\usepackage[utf8]{inputenc}
\usepackage[T1]{fontenc}
\usepackage{lmodern}
\usepackage{graphicx}
\usepackage{hyperref}
\usepackage{listings}
\usepackage{xcolor}
\usepackage{geometry}
\usepackage{titlesec}

% Page layout
\geometry{a4paper, margin=1in}

% Colors for links and listings
\definecolor{codegreen}{rgb}{0,0.6,0}
\definecolor{codegray}{rgb}{0.5,0.5,0.5}
\definecolor{codepurple}{rgb}{0.58,0,0.82}
\definecolor{backcolour}{rgb}{0.95,0.95,0.92}
\definecolor{linkcolor}{rgb}{0.1,0.1,0.65}

\hypersetup{
    colorlinks=true,
    linkcolor=linkcolor,
    urlcolor=blue,
    citecolor=black,
}

% Code listing style
\lstdefinestyle{mystyle}{
    backgroundcolor=\color{backcolour},   
    commentstyle=\color{codegreen},
    keywordstyle=\color{magenta},
    numberstyle=\tiny\color{codegray},
    stringstyle=\color{codepurple},
    basicstyle=\ttfamily\footnotesize,
    breakatwhitespace=false,         
    breaklines=true,                 
    captionpos=b,                    
    keepspaces=true,                 
    numbers=left,                    
    numbersep=5pt,                  
    showspaces=false,                
    showstringspaces=false,
    showtabs=false,                  
    tabsize=2
}
\lstset{style=mystyle}

% Title formatting
\titleformat{\section}
{\Large\bfseries}{\thesection}{1em}{}[{\titlerule[0.8pt]}]
\titleformat{\subsection}
{\large\bfseries}{\thesubsection}{1em}{}

% Title and Author
\title{\Huge \textbf{Application Development Tutorial: \\ [CartoonID]}}
\author{Author Name \\ \texttt{author@email.com}}
\date{\today}

\begin{document}

\maketitle

\begin{abstract}
This tutorial provides a comprehensive guide to setting up the development environment, installing necessary dependencies, and building the \textbf{[Your App Name Here]} application. By the end of this guide, you will have a fully functional development workspace ready for coding.
\end{abstract}

\tableofcontents
\newpage

\section{Introduction}
\subsection{About the Application}
[Provide a brief overview of the application you are developing. What is its purpose? What problem does it solve? What technology stack (e.g., web, mobile, desktop) does it use?]

\subsection{Who This Tutorial Is For}
This guide is intended for developers who are familiar with basic programming concepts and are looking to set up their machine for contributing to the \textbf{[Your App Name Here]} project.

\section{Development Machine Setup}
This section details the specifications and configuration of the host computer required for development.

\subsection{Recommended Host Specifications}
For an optimal development experience, we recommend the following hardware and operating system:

\begin{itemize}
    \item \textbf{OS:} Ubuntu 22.04 LTS / Windows 10+ / macOS Monterey 12.3+
    \item \textbf{CPU:} Quad-core processor (2.5 GHz or faster)
    \item \textbf{RAM:} 8 GB minimum (16 GB recommended)
    \item \textbf{Storage:} At least 20 GB of free disk space (for tools, SDKs, and dependencies)
    \item \textbf{Permissions:} Administrator/Sudo privileges to install software
\end{itemize}

\subsection{Required Base Software}
Before installing project-specific toolkits, ensure the following are installed on your system:
\begin{itemize}
    \item \textbf{Git} (for version control)
    \item \textbf{curl} or \textbf{wget} (for downloading files from the command line)
\end{itemize}

You can verify their installation with:
\begin{lstlisting}[language=bash]
  git --version
  curl --version
\end{lstlisting}

\section{Toolkits \& Core Development Tools}
This project relies on the following major development tools and SDKs. Follow the official links for installation instructions specific to your OS.

\subsection{Primary Toolkit: [e.g., Flutter SDK]}
\begin{description}
    \item[Version:] 3.13.0 (Stable channel)
    \item[Purpose:] The main framework used for building the cross-platform user interface.
    \item[Installation Guide:] \url{https://docs.flutter.dev/get-started/install}
\end{description}

\subsection{Programming Language: [e.g., Python]}
\begin{description}
    \item[Version:] Python 3.10+
    \item[Purpose:] Used for writing the backend logic and scripts.
    \item[Installer:] Download from the official Python website or use a version manager like \texttt{pyenv}.
\end{description}

\subsection{Integrated Development Environment (IDE)}
We recommend using \textbf{Visual Studio Code} with the following extensions:
\begin{itemize}
    \item [Extension Pack Name] (e.g., \textbf{Flutter} extension by Dart Code)
    \item [Dart] extension (if applicable)
    \item [Python] extension by Microsoft
\end{itemize}

Alternatively, you can use \textbf{Android Studio} or \textbf{IntelliJ IDEA} with the appropriate plugins.

\section{Project Dependencies}
The project's dependencies are managed by [e.g., pub, pip, npm]. They are listed in the configuration files and will be installed automatically.

\subsection{Configuration Files}
The main dependency files for this project are:
\begin{itemize}
    \item \texttt{pubspec.yaml} - (For Flutter/Dart packages)
    \item \texttt{requirements.txt} - (For Python packages)
    % Add more as needed, e.g., package.json, cargo.toml
\end{itemize}

\subsection{Installing Dependencies}
Clone the project repository and navigate into its directory:
\begin{lstlisting}[language=bash]
  git clone https://github.com/your-username/your-repo-name.git
  cd your-repo-name
\end{lstlisting}

Install all required dependencies by running the following commands:
\begin{lstlisting}[language=bash]
  # Example for a Flutter/Dart project
  flutter pub get

  # Example for a Python project
  pip install -r requirements.txt
\end{lstlisting}

\subsection{Key Dependency Overview}
Here are some of the most important direct dependencies this project uses:
\begin{itemize}
    \item %\texttt{http: ^0.13.5} - For making HTTP requests to our backend API.
    \item %\texttt{provider: ^6.0.5} - For state management within the app.
    \item  %\texttt{shared\_preferences: ^2.0.15} - For storing key-value data locally on the device.
    % Add your own key dependencies here
\end{itemize}

\section{Verifying Your Setup}
To ensure everything is installed correctly, run the following commands:

\begin{lstlisting}[language=bash]
  # Check toolkit versions
  flutter --version
  python --version
  dart --version

  # Run a basic check on the project
  flutter doctor
  flutter analyze
\end{lstlisting}

The output of \texttt{flutter doctor} is particularly important as it verifies that all required toolkits (like the Android SDK or Xcode) are properly installed and configured.

\section{Conclusion}
Your development environment for \textbf{[Your App Name Here]} is now ready! You should have all the necessary toolkits and dependencies installed. You can now proceed to build and run the application for the first time.

The next steps are usually:
\begin{enumerate}
    \item \textbf{Build the project:} \texttt{flutter build}
    \item \textbf{Run the project:} \texttt{flutter run}
    \item \textbf{Start coding!}
\end{enumerate}

For any issues during setup, please refer to the \texttt{README.md} file in the project repository or check the \textbf{Troubleshooting} section (if available).

\end{document}
